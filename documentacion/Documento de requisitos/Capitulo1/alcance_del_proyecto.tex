\section{Alcance del proyecto}

Al final del per�odo de desarrollo la herramienta debe contar con las siguientes prestaciones.

\begin{itemize}
    \item	El sistema permite la carga y la visualizaci�n de la red gr�ficamente.
    \item	El sistema solo resuelve dos clases de problemas de optimizaci�n, uno monoobjetivo y el otro multiobjetivo. El problema monoobjetivo ser� el de los costos de inversi�n para la construcci�n de las tuber�as. En cuanto al problema multiobjetivo, este ser� el de los costos energ�ticos y el n�mero de encendidos y apagado de las bombas.  
    \item	El sistema �nicamente cuenta con dos algoritmos implementados los cuales ser�n el Algoritmo Gen�tico (GA) y el algoritmo \textit{Non-Dominated Sorting Genetic Algorithm II} (NSGAII). El Algoritmo Gen�tico es usado para optimizar el problema monoobjetivo, mientras que NSGAII es aplicado para el multiobjetivo.
    \item	El sistema permite visualizar y guardar las soluciones de los algoritmos en un archivo.
    \item	El sistema permite que el usuario agregue nuevos algoritmos, operadores o problemas sin tener que modificar la interfaz de usuario.
\end{itemize}

Este proyecto no contempla la creaci�n de la red por lo que estas deben ser ingresadas como entradas al programa. 

Adem�s, esta herramienta �nicamente puede ser ocupada en equipos cuyo sistema operativo sea Windows de 64 bits y con la versi�n de Java 1.8 de 64bits debido a que se realizan llamadas a librer�as nativas compilada para sistema de 64bits.
