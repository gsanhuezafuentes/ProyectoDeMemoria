%% inicio, la clase del documento es iccmemoria.cls
\documentclass{especificacion}
\usepackage[onelanguage, commentsnumbered, linesnumbered, boxed, ruled]{algorithm2e}
\usepackage[draft=false,kerning=true]{microtype}
\usepackage{pdfpages}
\usepackage{amsfonts}
\usepackage{amssymb}
\usepackage{array}
\usepackage{longtable}
\usepackage{float}
\usepackage{especificacionutility}
\usepackage{setspace}
\usepackage{adjustbox}


%% datos generales y para la tapa
\titulo{Documento de especificaci�n de requisitos}
\subtitulo{Herramienta para la optimizaci�n de redes de distribuci�n de agua potable}
\author{Gabriel Gonzalo Alexander Sanhueza Fuentes}


%% inicio de documento
\begin{document}

%% crea la tapa
\maketitle

\adddevelopment{Gabriel Sanhueza Fuentes}{Administrador, Analista, Dise�ador, Implementador y Tester.}{gsanhueza15@alumnos~.utalca.cl}

\addcounterpart{Jimmy H. Guti�rrez-Bahamondes}{Cliente/Profesor gu�a}{}
\addcounterpart{Jimmy H. Guti�rrez-Bahamondes}{Cliente/Profesor co-gu�a}{}

\addrevision{0.1}{07/09/2019}{Primer borrador}
\addrevision{1}{15/06/2020}{Actualizaci�n documento. Agregados los requisitos faltantes.}


\RevisionHistoryPage


%% indices
\tableofcontents
%\listoffigures
%\listoftables

%% abstract

%WRITE HERE

\chapter{Introducci�n}
El proyecto que se desarrollara consiste en la creaci�n de una herramienta que haga uso de algoritmos metaheur�sticos para tratar y minimizar problemas existentes en redes de distribuci�n de agua potable.

Este proyecto solo abarcara dos problemas existentes dentro de las redes de distribuci�n de agua potable.

Para el desarrollo de este proyecto se usar�n librer�as ya existentes con el fin de reducir el tiempo de desarrollo. Estas librer�as son epanet2.dll, desarrollada en c, y JNA, librer�a en java para funciones nativas.


\section{Prop�sito del Sistema}

El proyecto consiste en el desarrollo de un sistema que permite simular y buscar soluciones a problemas presentes en las redes de distribuci�n de agua potable haciendo uso de algoritmos metaheur�sticos. Adicionalmente, el sistema es dise�ado de tal forma que pueda ser extendido a�adiendo nuevos problemas, algoritmos u operadores.
\section{Alcance del proyecto}

Al final del periodo de desarrollo la herramienta contara con las siguientes prestaciones.

\begin{itemize}
    \item	El sistema permitir� la carga y la visualizaci�n de la red gr�ficamente.
    \item	El sistema solo resolver� dos clases de problemas de optimizaci�n, uno mono-objetivo y el otro multiobjetivo. El problema mono-objetivo ser� el de los costos de inversi�n. En cuanto al problema multiobjetivo, este ser� el de los costos energ�ticos y el n�mero de encendidos y apagado de las bombas.  
    \item	El sistema �nicamente contara con dos algoritmos implementados los cuales ser�n el algoritmo gen�tico y NSGA-II. El algoritmo gen�tico ser� el usado para tratar el problema mono-objetivo, mientras que NSGA-II ser� aplicado al multiobjetivo.
    \item	El sistema permitir� visualizar y guardar las soluciones de los algoritmos en un archivo.
    \item	El sistema permitir� que el usuario agregue nuevos algoritmos, operadores o problemas sin tener que modificar la interfaz de usuario.
\end{itemize}

Este proyecto no contempla la creaci�n de la red por lo que estas deber�n ser ingresadas como entradas al programa. 

Adem�s, esta herramienta �nicamente podr� ser ocupada en equipos cuyo sistema operativo sea Windows debido a que se realizan llamadas a librer�as nativas.

\section{Contexto}

Este sistema ser� desarrollado utilizando el lenguaje de programaci�n java. Debido a que este es un lenguaje ampliamente utilizado y que cuenta con un gran soporte y comunidad que lo utilizan.

Como motor de c�lculo para llevar a cabo las simulaciones se utilizar� una librer�a desarrollada en c, que cuenta con funciones para llevar a cabo simulaciones de redes de agua potable. El nombre de esta librer�a es epanet2.dll. Las funciones que incorpora esta librer�a se encuentran explicadas en~\cite{Rossman2017}.

Desde lenguaje se realizar�n llamadas a librer�as nativas usando la librer�a JNA existente en java. Esta librer�a cuenta con las clases y m�todos necesarios para poder acoplar este sistema a la librer�a epanet2.dll desarrollada en c y que ser� usada como motor de c�lculo para llevar a cabo las simulaciones.
Puesto que una de las funcionalidades del sistema es permitir la ejecuci�n de algoritmos metaheur�sticos, se toma como base la arquitectura presentada por el framework JMetal.

JMetal es un framework para la optimizaci�n multiobjetivo con metaheur�sticas. Su arquitectura inicial~\cite{Durillo2010} involucraba una serie de problemas y dificultaban la realizaci�n de ciertas acciones que eran recurrentes. Adem�s, esta no hacia uso de las novedades incorporadas por Java como los gen�ricos. Es por esto, que posteriormente fue redise�ada, haciendo uso de patrones de dise�o, principios de la programaci�n orientada a objetos y aprovechando las caracter�sticas del lenguaje Java. Este redise�o se presenta en~\cite{Nebro2015}.

El contexto en el que se desenvolver� este sistema ser� en ambientes universitario, de investigaci�n y en el ambiente laboral. 

\section{Definiciones, Acr�nimos y Abreviaturas}

NSGA-II: Non-dominated Sorting Genetic Algorithm
%% genera las referencias
\bibliography{refs}
\chapter{Descripci�n General}
En esta secci�n se describir�n las caracter�sticas de los usuarios que har�n uso del sistema. Adem�s, se mencionar� el ambiente operacional de la soluci�n y la relaci�n que este proyecto tiene con otros proyectos. Finalmente, tambi�n se mencionar� las restricciones generales que existe en la realizaci�n de esta herramienta.

\section{Caracter�sticas de los Usuarios}

Este sistema solo cuenta con un tipo de usuario el cual tendr� acceso a todas las funcionalidades. Se espera que los usuarios que utilicen este sistema sean ingenieros, investigadores u personas que cuenten con el conocimiento b�sico acerca de redes de distribuci�n de agua potable y metaheur�sticas, que les servir� para interpretar los resultados generados por los algoritmos.
\section{Ambiente operacional de la soluci�n}

El ambiente operacional en el que se desarrolla el sistema es el siguiente:
\begin{itemize}
    \item Intel(R) Core(TM) i7-7700HQ CPU @ 2.80Ghz 2.8Ghz
    \item RAM 16GB DDR4
    \item HDD 7200rpm 1T
    \item SSD 256GB PCIe NVME M.2 
    \item Windows 10 x64
    \item NVIDIA GeForce GTX 1050
\end{itemize}


\section{Relaci�n con otros proyectos}

El sistema depende de la librer�a nativa epanet2\_64bit.dll, ya que usa esta librer�a como motor de c�lculo. La librer�a cuenta con 54 funciones dentro de las cuales se encuentran funciones para correr las simulaciones, modificar y obtener las configuraciones de la red, modificar los elementos que conforman la red y generar reportes.

Las llamadas desde Java a la librer�a nativa ser�n realizadas a trav�s de la librer�a \textit{Java Native Access} (JNA).

Adicionalmente, el sistema toma la arquitectura utilizada por el framework de optimizaci�n multiobjetivo Jmetal como base para agregar los algoritmos, operadores y problemas. Esta arquitectura ser� modificada seg�n se necesite para satisfacer los requisitos del sistema.

\section{Restricciones Generales}

\begin{itemize}
    \item La red ser� ingresada como entrada al programa a trav�s de un archivo .inp.
    \item La herramienta solo estar� disponible para el sistema operativo Window.
\end{itemize}




\chapter{Requisitos}
En esta cap�tulo se presentan los requisitos capturados, los cuales est�n sujetos a cambios a medida que se avanza en las iteraciones.

\section{Requisitos de usuario}

A continuaci�n, se presentar�n los requisitos de usuarios que han sido obtenidos para el desarrollo de este proyecto

Durante la presentaci�n de estos requisitos se hace referencia al archivo de configuraci�n de red. �ste debe ser generado utilizando la aplicaci�n Epanet y guardado con la extensi�n ``inp'' a partir de ahora nos referiremos al archivo de configuraci�n de red simplemente como inp o archivo inp.

\begin{requisito}
    \Requisito{RU001}{Cargar una red.}
    \Descripcion{La red que es representada por el archivo .inp debe ser cargada en el programa para poder llevar a cabo la simulaci�n.}
    
    \Fuente{Jimmy Guti�rrez}
    \Prioridad{Alta}
    \Estabilidad{Intransable}
    \FechaA{09/09/2019}
    \Estado{Cumple}
    \Incremento{1}
    \Tipo{Funcional}
\end{requisito}
    
\begin{requisito}
    \Requisito{RU002}{Resolver el problema monoobjetivo (\textit{Pipe Optimizing})  usando el Algoritmo Gen�tico.}
    \Descripcion{El Algoritmo Gen�tico debe ser aplicado para resolver el problema monoobjetivo que tiene como funci�n objetivo el costo de inversi�n y como variable de decisi�n el di�metro de las tuber�as.}
    
    \Fuente{Jimmy Guti�rrez}
    \Prioridad{Alta}
    \Estabilidad{Intransable}
    \FechaA{09/09/2019}
    \Estado{Cumple}
    \Incremento{2}
    \Tipo{Funcional}
\end{requisito}
    
\begin{requisito}
    \Requisito{RU003}{Resolver el problema multiobjetivo (\textit{Pumping Scheduling}) usando el Algoritmo NSGAII.}
    \Descripcion{El algoritmo NSGAII debe ser aplicado al problema multiobjetivo cuyas funciones a optimizar son los costos energ�ticos y el n�mero de encendido y apagado de las bombas (\textit{Pumping Schedule}).}
    
    \Fuente{Jimmy Guti�rrez}
    \Prioridad{Alta}
    \Estabilidad{Intransable}
    \FechaA{09/09/2019}
    \Estado{Cumple}
    \Incremento{4}
    \Tipo{Funcional}
\end{requisito}
    
\begin{requisito}
    \Requisito{RU004}{Visualizar red en una interfaz gr�fica.}
    \Descripcion{Se debe mostrar en la interfaz gr�fica una representaci�n de la red (Un dibujo, etc) generada a partir de la informaci�n contenida en el archivo inp.}
    
    \Fuente{Jimmy Guti�rrez}
    \Prioridad{Moderada}
    \Estabilidad{Intransable}
    \FechaA{09/09/2019}
    \Estado{Cumple}
    \Incremento{3}
    \Tipo{Funcional}
\end{requisito}
    
\begin{requisito}
    \Requisito{RU005}{Exportar los resultados de los algoritmos aplicados en dos archivos, uno para las variables y otro para los objetivos.}
    \Descripcion{Se deben poder exportar las soluciones generados por la ejecuci�n de los algoritmos sobre un problema a un conjunto de archivos. Espec�ficamente, serian 2 archivos. El primer archivo debe guardar las variables de las soluciones mientras que el segundo archivo los objetivos de las soluciones.}
    
    \Fuente{Jimmy Guti�rrez}
    \Prioridad{Moderada}
    \Estabilidad{Intransable}
    \FechaA{09/09/2019}
    \Estado{Cumple}
    \Incremento{3}
    \Tipo{Funcional}
\end{requisito}
    
\begin{requisito}
    \Requisito{RU006}{Implementar el operador IntegerSBXCrossover.}
    \Descripcion{El operador IntegerSBXCrossover es uno de los operadores de cruzamiento. Este operador en base a c�lculos probabil�sticos combina dos soluciones para crear dos nuevas soluciones hijas.}
    
    \Fuente{Jimmy Guti�rrez}
    \Prioridad{Alta}
    \Estabilidad{Intransable}
    \FechaA{14/10/2019}
    \Estado{Cumple}
    \Incremento{2}
    \Tipo{Funcional}
\end{requisito}
    
\begin{requisito}
    \Requisito{RU007}{Implementar el operador IntegerSinglePointCrossover.}
    \Descripcion{El operador IntegerSinglePointCrossover es un operador de cruzamiento. Viendo la soluci�n como un vector, este operador toma dos soluciones y elige un punto a partir del cual los valores de una soluci�n se intercambiar�n con los valores de la otra soluci�n. Este operador usa una probabilidad de cruzamiento y solamente realiza el intercambio de los valores en la soluci�n cuando un n�mero generado aleatoriamente es menor que la probabilidad de cruzamiento.}
    
    \Fuente{Jimmy Guti�rrez}
    \Prioridad{Alta}
    \Estabilidad{Intransable}
    \FechaA{14/10/2019}
    \Estado{Cumple}
    \Incremento{2}
    \Tipo{Funcional}
\end{requisito}
    
\begin{requisito}
    \Requisito{RU008}{Implementar el operador IntegerPolynomialMutation.}
    \Descripcion{El operador IntegerPolynomialMutation es un operador de mutaci�n. Este operador de mutaci�n usa c�lculos probabil�sticos para mutar algunos variables de decisi�n que forman parte de la soluci�n.}
    
    \Fuente{Jimmy Guti�rrez}
    \Prioridad{Alta}
    \Estabilidad{Intransable}
    \FechaA{14/10/2019}
    \Estado{Cumple}
    \Incremento{2}
    \Tipo{Funcional}
\end{requisito}
    
\begin{requisito}
    \Requisito{RU009}{Implementar el operador IntegerSimpleRandomMutation.}
    
    \Descripcion{El operador IntegerSimpleRandomMutation es un operador de mutaci�n. Este operador muta una variable de decisi�n cuando un n�mero generado aleatoriamente es menor que la probabilidad de mutaci�n establecida. El operador recorre cada variable de decisi�n realizando lo descrito anteriormente. La mutaci�n mencionada por este operador consiste en cambiar el valor de la variable de decisi�n por otro valor aleatorio. }
    
    \Fuente{Jimmy Guti�rrez}
    \Prioridad{Alta}
    \Estabilidad{Intransable}
    \FechaA{14/10/2019}
    \Estado{Cumple}
    \Incremento{2}
    \Tipo{Funcional}
\end{requisito}
    
    
\begin{requisito}
    \Requisito{RU010}{Implementar el operador IntegerRangeRandomMutation.}
    
    \Descripcion{El operador IntegerRangeRandomMutation es un operador de mutaci�n. Este operador muta una variable de decisi�n cuando un n�mero generado aleatoriamente es menor que la probabilidad de mutaci�n establecida. El operador recorre cada variable de decisi�n realizando lo descrito anteriormente. La mutaci�n realizada por este operador consiste en cambiar el valor de la variable de decisi�n por otro valor aleatorio que se encuentre entre un rango establecido.
    \singlespacing
    Ejemplo:\singlespacing

    Variable de decisi�n: 3\\
    Rango: 2\\
    La variable de decisi�n despu�s de aplicado el operador puede tomar un valor entre [1, 5].
    }
    \Fuente{Jimmy Guti�rrez}
    \Prioridad{Alta}
    \Estabilidad{Intransable}
    \FechaA{14/10/2019}
    \Estado{Cumple}
    \Incremento{2}
    \Tipo{Funcional}
\end{requisito}
    
    
\begin{requisito}
    \Requisito{RU011}{Implementar el operador UniformSelection.}
    \Descripcion{El operador UniformSelection~\cite{Iglesias-2004} es un operador de selecci�n. Este operador de selecci�n ordena la poblaci�n y asigna una probabilidad m�xima y m�nima a la mejor y peor soluci�n respectivamente. A las soluciones que se encuentran entre la mejor y la peor soluci�n se le asigna una probabilidad entre el m�ximo y m�nimo obtenido anteriormente. Si la probabilidad de la soluci�n es mayor a 1.5 entonces la soluci�n se duplica en la nueva poblaci�n. Si la probabilidad esta entre 0.5 y 1.5, entonces en la nueva poblaci�n se agrega la soluci�n solo una vez. Las soluciones cuya probabilidad es menor que 0.5 no aparecen en la nueva poblaci�n.
    
    La probabilidad m�xima puede se calcula como $p_{max} = \beta/N_c$ mientras que la probabilidad m�nima se calcula de acuerdo a $p_{min} = (2-\beta)/N_c$. Donde $\beta$ es un numero entre 1.5 y 2, y $N_c$ es el n�mero de soluciones presentes en el conjunto sobre el que se realizar� la selecci�n.

    La probabilidad de las soluciones intermedias se calcula de acuerdo a $$p_i=p_{min}+(p_{max}-p_{min})\times((N_c-i)/(N_c-1))$$}
    \Fuente{Jimmy Guti�rrez}
    \Prioridad{Alta}
    \Estabilidad{Intransable}
    \FechaA{14/10/2019}
    \Estado{Cumple}
    \Incremento{2}
    \Tipo{Funcional}
\end{requisito}
    
\begin{requisito}
    \Requisito{RU012}{Crear archivo inp de la soluci�n generada.}
    \Descripcion{Al ejecutar un algoritmo metaheur�stico este devuelve una o un conjunto de soluciones. A partir de alguna de estas soluciones se debe crear un archivo inp en el que se vean reflejados los resultados de la soluci�n.}
    
    \Fuente{Jimmy Guti�rrez}
    \Prioridad{Moderada}
    \Estabilidad{Intransable}
    \FechaA{15/10/2019}
    \Estado{Cumple}
    \Incremento{3}
    \Tipo{Funcional}
\end{requisito}
    
\begin{requisito}
    \Requisito{RU013}{Mostrar las soluciones de los algoritmos en la interfaz de usuario.}
    \Descripcion{Mostrar los resultados de la ejecuci�n del algoritmo, es decir, las variables y los valores objetivos resultantes. Adicionalmente, �stas deben poder ser guardadas en el equipo del usuario. Las variables de decisi�n se guardar�n en un archivo (VAR), mientras que los valores de los objetivos se guardar�n en otro (FUN).}
    
    \Fuente{Jimmy Guti�rrez}
    \Prioridad{Alta}
    \Estabilidad{Transable}
    \FechaA{30/11/2019}
    \Estado{Cumple}
    \Incremento{3}
    \Tipo{Funcional}
\end{requisito}
%% hasta aki llevo revisado la ortograf�a 
    
\begin{requisito}
    \Requisito{RU014}{Mostrar las caracter�sticas de la red.}
    \Descripcion{Mostrar las caracter�sticas que posee la red. Esto puede ser realizado cuando se presiona el elemento de la red o agregando alg�n componente que muestre los elementos que conforman la red.}
    
    \Fuente{Jimmy Guti�rrez}
    \Prioridad{Moderada}
    \Estabilidad{Transable}
    \FechaA{30/11/2019}
    \Estado{Cumple}
    \Incremento{3}
    \Tipo{Funcional}
\end{requisito}
    
    
\begin{requisito}
    \Requisito{RU015}{Graficar las soluciones.}
    \Descripcion{Mostrar en un plano cartesiano las soluciones que se van obteniendo a medida que se ejecuta el algoritmo. Solo considerar hasta 2 objetivos.}
    
    \Fuente{Jimmy Guti�rrez}
    \Prioridad{Baja}
    \Estabilidad{Transable}
    \FechaA{30/11/2019}
    \Estado{Cumple}
    \Incremento{3}
    \Tipo{Funcional}
\end{requisito}
    
\begin{requisito}
    \Requisito{RU016}{Hacer el programa f�cil de extender.}
    \Descripcion{El programa debe poder f�cilmente agregar nuevos algoritmos, operadores y problemas.}
    
    \Fuente{Jimmy Guti�rrez}
    \Prioridad{Alta}
    \Estabilidad{Transable}
    \FechaA{30/11/2019}
    \Estado{Cumple}
    \Incremento{3}
    \Tipo{No Funcional}
\end{requisito}
    
\begin{requisito}
    \Requisito{RU017}{Mostrar en la interfaz de usuario los problemas, agrupando los algoritmos que pueden ser utilizados para resolverlos.}
    \Descripcion{En el men� donde se muestran los problemas que pueden ser resueltos debe aparecer el nombre del problema y en este se deben agrupar los algoritmos que pueden ser utilizados para resolverlo.
    \singlespacing
    EJ:
    \singlespacing
    Pumping Scheduling \\
    $>$ NSGAII\\
    $>$ SPA2

    }
    \Fuente{Daniel Mora-Meli�}
    \Prioridad{Baja}
    \Estabilidad{Intransable}
    \FechaA{27/01/2020}
    \Estado{Cumple}
    \Incremento{5}
    \Tipo{Funcional}
\end{requisito}
    
\begin{requisito}
    \Requisito{RU018}{Permitir realizar m�ltiples simulaciones independientes para resolver el problema multiobjetivo.}
    \Descripcion{Durante la resoluci�n del problema multiobjetivo se debe poder escoger cuantas veces se aplicar� el algoritmo, independientemente uno de otros. La soluci�n final ser� la Frontera de Pareto del conjunto formado por todas las soluciones generadas a partir de cada ejecuci�n independiente del algoritmo. Este concepto se conoce en algunos framework como Experimentos.}
    
    \Fuente{Jimmy Guti�rrez}
    \Prioridad{Alta}
    \Estabilidad{Intransable}
    \FechaA{27/01/2020}
    \Estado{Cumple}
    \Incremento{5}
    \Tipo{Funcional}
\end{requisito}
    
\begin{requisito}
    \Requisito{RU019}{Guardar los resultados temporales por cada simulaci�n independiente del problema multiobjetivo y generar los archivos al final de todas las simulaciones con los mejores resultados obtenidos.}
    \Descripcion{Las simulaciones multiobjetivos permiten realizar m�ltiples simulaciones independientes, entregando cada ejecuci�n como resultado el conjunto de soluciones que conforman su frontera de Pareto. Sin embargo, una vez se terminan todas las repeticiones y se genera la soluci�n final, se pierden los resultados intermedios de cada repetici�n. Es por ello, que mientras se van terminando cada simulaci�n independiente, se debe guardar un respaldo de los resultados de cada repetici�n del algoritmo.}
    
    \Fuente{Jimmy Guti�rrez}
    \Prioridad{Baja}
    \Estabilidad{Intransable}
    \FechaA{27/01/2020}
    \Estado{Cumple}
    \Incremento{5}
    \Tipo{Funcional}
\end{requisito}
    
\begin{requisito}
    \Requisito{RU020}{Permitir realizar simulaciones hidr�ulicas utilizando los valores por defectos que vienen en el archivo .inp y visualizar los resultados.}
    \Descripcion{Utilizando los valores que vienen por defecto en el archivo inp se debe poder llevar a cabo la simulaci�n hidr�ulica de la red. Posteriormente, los resultados podr�n ser visualizados por el usuario.}
    
    \Fuente{Daniel Mora-Meli�}
    \Prioridad{Alta}
    \Estabilidad{Intransable}
    \FechaA{27/01/2020}
    \Estado{Cumple}
    \Incremento{5}
    \Tipo{Funcional}
\end{requisito}
    
\begin{requisito}
    \Requisito{RU021}{Agregar el algoritmo multiobjetivo SMPSO.}
    \Descripcion{Con el fin de comprobar que se pueden acoplar nuevos algoritmos se solicita por parte del usuario que se agregue el algoritmo SMPSO. }
    
    \Fuente{Jimmy Guti�rrez}
    \Prioridad{Baja}
    \Estabilidad{Intransable}
    \FechaA{13/05/2020}
    \Estado{Cumple}
    \Incremento{6}
    \Tipo{Funcional}
\end{requisito}
    
\begin{requisito}
    \Requisito{RU022}{Agregar el algoritmo multiobjetivo SPA2.}
    \Descripcion{Con el fin de comprobar que se pueden acoplar nuevos algoritmos se solicita por parte del usuario que se agregue el algoritmo SPA2. }
    
    \Fuente{Jimmy Guti�rrez}
    \Prioridad{Baja}
    \Estabilidad{Intransable}
    \FechaA{13/05/2020}
    \Estado{Cumple}
    \Incremento{6}
    \Tipo{Funcional}
\end{requisito}
    
\begin{requisito}
    \Requisito{RU023}{Incluir en el dibujo de la red s�mbolos para diferencias los elementos que componen la red.}
    \Descripcion{Se requiere que se agreguen en la representaci�n gr�fica de la red s�mbolos para distinguir los distintos elementos que la componen.}
    
    \Fuente{Jimmy Guti�rrez}
    \Prioridad{Moderada}
    \Estabilidad{Intransable}
    \FechaA{13/05/2020}
    \Estado{Cumple}
    \Incremento{6}
    \Tipo{Funcional}
\end{requisito}
    
\begin{requisito}
    \Requisito{RU024}{Permitir realizar m�ltiples simulaciones independientes para resolver el problema monoobjetivo.}
    \Descripcion{Durante la resoluci�n del problema monoobjetivo se debe poder escoger cuantas veces se repetir� el algoritmo, siendo independiente una repetici�n de la otra. Al finalizar cada repetici�n se solicita mostrar los resultados de cada repetici�n.}
    
    \Fuente{Jimmy Guti�rrez}
    \Prioridad{Alta}
    \Estabilidad{Intransable}
    \FechaA{13/05/2020}
    \Estado{Cumple}
    \Incremento{6}
    \Tipo{Funcional}
\end{requisito}
    
\begin{requisito}
    \Requisito{RU025}{Agregar un men� de configuraci�n.}
    \Descripcion{Agregar un men� de configuraciones donde poder establecer el tama�o de los puntos, si mostrar o no el grafico de resultados y limitar el n�mero de resultados retornados por el problema multiobjetivo.}
    
    \Fuente{Jimmy Guti�rrez}
    \Prioridad{Baja}
    \Estabilidad{Intransable}
    \FechaA{13/05/2020}
    \Estado{Cumple}
    \Incremento{6}
    \Tipo{Funcional}
\end{requisito}
    
\begin{requisito}
    \Requisito{RU026}{Usar en el gr�fico, para cada repetici�n del algoritmo en un experimento, un color distinto.}
    \Descripcion{Para facilitar la distinci�n entre cada repetici�n de un algoritmo en un experimento se solicita variar que se cambie el color mostrado para cada repetici�n.}
    
    \Fuente{Jimmy Guti�rrez}
    \Prioridad{Moderada}
    \Estabilidad{Transable}
    \FechaA{13/05/2020}
    \Estado{Cumple}
    \Incremento{6}
    \Tipo{Funcional}
\end{requisito}
    
\begin{requisito}
    \Requisito{RU027}{Mostrar una leyenda que pueda ser activada y desactivada con los s�mbolos mostrados sobre el dibujo de la red.}
    \Descripcion{Se requiere que en la ventana de representaci�n de la red se muestre una leyenda en la que se presenten los s�mbolos utilizados por la aplicaci�n para representar cada componente de la red.}
    
    \Fuente{Jimmy Guti�rrez}
    \Prioridad{Moderada}
    \Estabilidad{Intransable}
    \FechaA{13/05/2020}
    \Estado{Cumple}
    \Incremento{6}
    \Tipo{Funcional}
\end{requisito}
    
    
\begin{requisito}
    \Requisito{RU028}{Mostrar en la ventana de configuraci�n informaci�n sobre el problema, como los objetivos, el nombre del algoritmo a utilizar y el nombre del problema que se quiere resolver.}
    \Descripcion{Al momento de querer realizar una optimizaci�n no se entrega suficiente informaci�n acerca de los objetivos del problema. Es por esto, que se requiere poder agregar alguna descripci�n que pueda ser visualizada cuando se abra la ventana de configuraci�n del experimento.}
    
    \Fuente{Jimmy Guti�rrez}
    \Prioridad{Moderada}
    \Estabilidad{Intransable}
    \FechaA{13/05/2020}
    \Estado{Cumple}
    \Incremento{6}
    \Tipo{Funcional}
\end{requisito}
    
\begin{requisito}
    \Requisito{RU029}{A�adir en la ventana de resultados del problema columnas extras que muestren las configuraciones utilizadas con el problema.}
    \Descripcion{Se solicita que se pueda agregar mas informaci�n a la ventana de resultados mostrada cuando se termina una optimizaci�n. Se propone usar un mapa con los valores adicionales que se quieran mostrar, en donde la clave sea el nombre de la columna y el valor sea lo mostrado en la celda.}
    
    \Fuente{Jimmy Guti�rrez}
    \Prioridad{Moderada}
    \Estabilidad{Intransable}
    \FechaA{13/05/2020}
    \Estado{Cumple}
    \Incremento{6}
    \Tipo{Funcional}
\end{requisito}
    
    
\begin{requisito}
    \Requisito{RU030}{Exportar los resultados de los algoritmos aplicados como un Excel.}
    \Descripcion{Se requiere exportar la table de resultados de la optimizaci�n del algoritmo a un archive excel.}
    
    \Fuente{Jimmy Guti�rrez}
    \Prioridad{Moderada}
    \Estabilidad{Intransable}
    \FechaA{13/05/2020}
    \Estado{Cumple}
    \Incremento{6}
    \Tipo{Funcional}
\end{requisito}
    
\begin{requisito}
    \Requisito{RU031}{Exportar el gr�fico utilizado para mostrar visualmente las soluciones a una imagen (PNG o SVG)}
    \Descripcion{Se requiere exportar el gr�fico de resultados mostrado durante la ejecuci�n de un experimento, ya sea para el problema monoobjetivo como el multiobjetivo. Se requiere idealmente SVG, en caso de que este formato no sea posible se acepta la utilizaci�n de PNG.}
    
    \Fuente{Jimmy Guti�rrez}
    \Prioridad{Moderada}
    \Estabilidad{Transable}
    \FechaA{13/05/2020}
    \Estado{Cumple}
    \Incremento{6}
    \Tipo{Funcional}
\end{requisito}
    
\begin{requisito}
    \Requisito{RU032}{Permitir indicar valores por defecto a los operadores y a los problemas.}
    \Descripcion{Se solicita que se puedan ingresar valores por defectos que puedan ser visualizados en la ventana de configuraci�n del problema antes de llevar a cabo la resoluci�n del algoritmo. Los valores por defecto deben ser agregados donde se esperen recibir valores num�ricos.}
    
    \Fuente{Jimmy Guti�rrez}
    \Prioridad{Moderada}
    \Estabilidad{Intransable}
    \FechaA{13/05/2020}
    \Estado{Cumple}
    \Incremento{6}
    \Tipo{Funcional}
\end{requisito}
    
\input{Capitulo3/rs}
\section{Matriz de Trazado Requisitos de Usuario vs. Requisitos de Software}
La matriz de trazabilidad de los requisitos de usuario y de sistema que se presenta a continuaci�n permite ver la relaci�n y dependencia que un requisito de sistema tiene con los requisitos de usuario.

\begin{figure}[H]
    \centering
    \adjincludegraphics[width=\textwidth, trim={0 0 0 {0.3\height}},clip, rotate = 180]{Capitulo3/assets/matriz_req-u_req-s.eps}
\end{figure}

\begin{figure}[H]
    \centering
    \adjincludegraphics[width=\textwidth, trim={0 {0.7\height} 0 0},clip, rotate = 180]{Capitulo3/assets/matriz_req-u_req-s.eps}
      \caption{Matriz de requisito de usuario versus requisitos de sistema.}
      \label{fig:matriz_req}
\end{figure}

%% ambiente glosario
%\begin{glosario}
%  \item[RDA] Este es el significado del primer t�rmino, realmente no se bien lo que significa pero podr�a haberlo averiguado si hubiese tenido un poco mas de tiempo.
%  \item[GA] Este si se lo que significa pero me da lata escribirlo...
%\end{glosario}


%% genera las referencias
%\bibliography{refs}


%% comienzo de la parte de anexos
%\appendixpart

%% contenido del primer anexo
%%\appendix{Apendix}

\end{document}

   

