\section{Presentaci�n del problema}

Los encargados de implementar sistemas de distribuci�n de agua potable, no cuentan con suficientes herramientas y tiempo para su correcta gesti�n. Por lo tanto, no es posible utilizar los recursos asociados de forma eficiente. Adem�s, las herramientas existentes no satisfacen las necesidades de usabilidad y costo, debido a que son poco intuitivas y de pago.

El escoger las especificaciones de una red de agua potable es una tarea dif�cil debido a que hay que evaluar el rendimiento general del sistema en funci�n de un conjunto de variables que se mueven en un rango muy elevado de posibilidades. Debido a esto, el uso de herramientas que optimicen la selecci�n de estas caracter�sticas puede ayudar considerablemente a reducir costos operaciones y de inversi�n.

Finalmente, es importante destacar que la construcci�n de un sistema que permita realizar la optimizaci�n de RDA es compleja. Necesita del conocimiento t�cnico de expertos en el �rea de hidr�ulica y computaci�n. Involucra el trabajo con programas de simulaci�n computacional que modelan las caracter�sticas de los sistema de agua bajo presi�n y de algoritmos metaheur�sticos que los subordinen.   

