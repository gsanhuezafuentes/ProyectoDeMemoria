\section{Herramientas para la implementaci�n del software}
\subsubsection{Java}
Java es un lenguaje de programaci�n de alto nivel orientado a objetos y de prop�sito general. Un programa java se ejecuta sobre la maquina virtual llamada la Java Virtual Machine, la cual le da a este lenguaje la caracter�stica de ser multiplataforma. Adicionalmente, java incorpora el soporte para multi-hilos, una poderosa herramienta que permite la ejecuci�n de distintas instrucciones de c�digo al mismo tiempo \cite{Gosling2015}. Ademas. este lenguaje tambi�n incorpora una caracter�stica conocida como el recolector de basura, el cual se encarga de limpiar la memoria de objetos que ya no est�n siendo utilizados.  Fue anunciado por Sun Microsystems en Mayo de 1995 \cite{3java}. 

\subsubsection{Java Reflection}
Caracter�stica de java que permite que un programa se auto examine. Esta caracter�stica est� disponible a trav�s de la Java Reflection API, la cual cuenta con m�todos para obtener los metadatos de las clases, m�todos, constructores, campos o par�metros. Esta API tambi�n permite crear nuevos objetos cuyo tipo era desconocido al momento de compilar el programa \cite{Braux1999}.

\subsubsection{Java Annotation}
Caracter�stica de java para agregar metadatos a elementos de java (clases, m�todos, par�metros, etc.) [7]. Las anotaciones no tienen efecto directo sobre el c�digo, pero cuando son usadas junto con otras herramientas pueden llegar a ser muy �tiles. Estas herramientas pueden analizar estas anotaciones y realizar acciones en base a estas, por ejemplo, generar archivos adicionales como clases de java, archivos xml, ser analizadas durante la ejecuci�n del programa v�a Java Reflection, para crear objetos cuyo tipo no conocemos en tiempo de compilaci�n; etc. 