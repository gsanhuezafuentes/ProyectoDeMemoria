\section{Concepci�n del proyecto}

Este proyecto se origina como una propuesta por parte del profesor del departamento de Ingenier�a en Obras Civiles, Daniel Mora Melia. �l, junto a un grupo de expertos de diversas �reas, presentaron y publicaron un articulo del proyecto JHawanet~\cite{JHawanet-2019}. Dicho articulo, presenta la integraci�n de dos librer�as independientes, JMetal y Epanet, como herramienta para llevar a cabo optimizaciones sobre RDA. JMetal~\cite{Durillo2010} es un Framework de Java, orientado a la optimizaci�n multiobjetivo y es usado como motor de optimizaci�n. Mientras que Epanet~\cite{Rossman1999} es una herramienta la cual permite realizar simulaciones en redes de agua potable.

Debido al articulo anteriormente mencionado, surgi� la idea de crear una herramienta gr�fica con el fin de facilitar la optimizaci�n de redes de agua potable. Puesto que la utilizaci�n de la herramienta JHawanet requiere conocimiento computacional avanzado. Y de esta forma, permitir el trabajo de Ingenieros hidr�ulicos en un entorno especialmente dise�ado para su �rea sin perder la capacidad que la herramienta posee para que usuarios avanzados puedan incorporar y evaluar nuevos problemas y algoritmos. 

