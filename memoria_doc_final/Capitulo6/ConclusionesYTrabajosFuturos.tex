\chapter{Conclusiones Y Trabajos Futuros}
En este cap�tulo se presentan las conclusiones del trabajo desarrollado asi como la propuestas de trabajo futuro que puede ser realizados sobre la herramienta.

\section{Conclusiones}
Se concluyen que los objetivos definidos para este trabajo son exitosamente logrados. Adicionalmente, la metodolog�a de desarrollo escogida fue la correcta dando como resultado documentaci�n extra que puede ser utilizado por aquellos que deseen continuar con este proyecto para futuros trabajos. Estos documentos son la especificacion de requisitos (Anexo~\ref{appendix:requisito}) y dise�o (Anexo~\ref{appendix:diseno}); manual de usuario (Anexo~\ref{appendix:manual}); y la especificaci�n de casos de prueba (Anexo~\ref{appendix:prueba}).

Durante la aplicaci�n de la metodolog�a si bien hubieron algunos retrasos en su ejecuci�n se logro implementar todas las funcionalidades solicitadas por los interesados. Adicionalmente, la utilizaci�n de \textit{Java Reflection} y \textit{Java Annotation} para hacer la aplicaci�n extensible con el fin de agregar nuevos algoritmos, problemas y operadores fue de gran importancia.

En relaci�n a la evaluaci�n de la soluci�n se concluye que la aplicaci�n incorpora las funcionalidades solicitadas en un principio por los interesados y que �stas son de utilidad para los usuarios que se desempe�an en el �rea de redes de agua potable. Sin embargo, el manual de usuario generado deja que desear y debe ser mejorado.

\section{Trabajo futuro}
Una vez terminado el desarrollo del proyecto se identifico por parte del desarrollador, los colaboradores y los sujetos del estudio realizado, una serie de cadencias y detalles que pueden ser mejorados, as� como algunas funcionalidades extras que incrementan el valor de la aplicaci�n. �stas se listan a continuaci�n:

\begin{itemize}
    \item Agregar nuevos algoritmos, operadores y problemas al sistema.
    \item Recortar el n�mero de decimales con que se presentan los resultados en la aplicaci�n.
    \item Exportar el gr�fico de la red como una imagen vectorial.
    \item Exportar el gr�fico de soluciones para uno y dos objetivos como una imagen vectorial.
    \item Cambiar el tama�o de los iconos de la red cuando se posicione el cursor sobre �stos.
    \item Permitir agregar formulas utilizando Latex en la descripci�n de los algoritmos.
    \item Permitir utilizar distintos algoritmos en un mismo experimento.
    \item Incorporar m�tricas de comparaci�n entre algoritmos.
    \item Mostrar en la interfaz los patrones de demanda y bombeo de la red cuanto �sta los especifique.
    \item Permitir restablecer los valores por defecto en la pesta�a de configuraci�n del problema.
    \item Incorporar una cola de trabajo para ejecutar m�s de una optimizaci�n a la vez.
\end{itemize}


